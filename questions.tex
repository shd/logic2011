\documentclass[12pt,a4paper,oneside]{book}
\usepackage[utf8]{inputenc}
\usepackage[english,russian]{babel}
\begin{document}

\begin{center}
\begin{Large}Программа курса <<Математическая логика>>\end{Large}\\
ИТМО, группы 2537-2539, осень 2013 г.
\end{center}

\begin{enumerate}
\item Исчисление высказываний. Общезначимость, доказуемость и выводимость. Теорема о дедукции для исчисления высказываний.
\item Теорема о полноте исчисления высказываний.
\item Интуиционистское исчисление высказываний. Топологическая интерпретация исчисления, модели Крипке.
\item Исчисление предикатов. Общезначимость и выводимость. Теорема о дедукции в исчислении предикатов.
\item Теорема о полноте исчисления предикатов.
\item Теории первого порядка, структуры и модели. Аксиоматика Пеано. Формальная арифметика. 
\item Рекурсивные функции и отношения. Существование рекурсивных функций,
не являющихся примитивно-рекурсивными. Функция Аккермана.
\item Представимость функций в формальной арифметике. Бета-функция Гёделя. 
Представимость рекурсивных функций в формальной арифметике.
\item Выразимость отношений. Гёделева нумерация. Выводимость и рекурсивные функции.
\item Непротиворечивость и $\omega$-непротиворечивость. Первая теорема Гёделя о неполноте арифметики.
\item Вторая теорема Гёделя о неполноте арифметики, идея доказательства. 
$Consis$, условия выводимости Гильберта-Бернайса, их необходимость.
\item Теория множеств. Аксиоматика Цермело-Френкеля.
\item Вполне упорядоченные множества. Ординальные числа. Операции над ординальными числами. 
\item Теорема о непротиворечивости формальной арифметики.
\item Кардинальные числа. Теорема Лёвенгейма-Сколема. Парадокс Сколема.
\end{enumerate}

\end{document}